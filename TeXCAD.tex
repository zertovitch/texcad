\documentclass[11pt,a4paper]{article}
%
\usepackage{a4wide}
%
\usepackage[dvipdfm]{hyperref} %%% 18-Apr-2001 : hyperliens PDF - prudence!
%
%\oddsidemargin  =    -1cm
%\evensidemargin =    -1cm
%\textwidth      =    17cm
%\topmargin      =     0cm
%\textheight     =    23cm
%
\begin{document}
%
\parindent 0pt
%
\def\TC{{\TeX}CAD}
%
\begin{center}
% \input{tclogo1.tcp}
{\sf {\Huge \TC} {\LARGE User's manual}}
\end{center}
%
\def\bs{$\backslash$}
\def\hrefid#1{\href{#1}{{\tt {\small #1}}}}
%
$$ $$
%
\section{About {\TC}}
%
  {\TC} is a program for drawing or retouching \{picture\}s in \LaTeX.
  The features of environment \{picture\} are quite limited, but it
  presents the great advantage of requiring {\bf no} add-on.
  You also enjoy the same fonts, macros, formulas as elsewhere
  in your \LaTeX \, document
  \footnote{This holds for previewing too: from {\TC} version 4.2,
  you can insert your own definitions for previewing.}.
  \\
  In addition, {\TC} extends the original \{picture\} capabilities,
  even without any obligatory \LaTeX \, package, class or style sheet.\\
  Furthermore, you can still switch on some supported add-ons, but are not obliged to.\\

%
%  NB: the present document has hyperlinks; if you pass {\TC}.dvi through
%  {\tt dvipdfm}, the PDF file will provide them.

\subsection{Licence and distribution}
%
    {\TC} is free software and is distributed, including its sources,
    under the GNU General Public License.
    Copyright \copyright \, 2003..2012
    \href{http://www.fsf.org/}{Free Software Foundation.}
    Please read {\TC}.txt for Copyrights details.
%
\section{Presentation of {{\TC}} --- a few features}
%
{{\TC}} is rather thought for --- but
not limited to --- simple figures drawn by hand.
It lifts the annoyance of using a separate program
just for drawing a little diagram with a box and a few arrows.
The usual solution is
to use a specific program, save the picture in EPS format, rescale,
struggle to obtain
the \LaTeX\, fonts, set the formulas in the picture, make it appear
in the final format, which is never given. All this effort is a
massive loss of time and nerves.

\subsection{Files, I/O} The principle of {\TC} is very simple:
there is only one format, which is \LaTeX\, code.
This means that you don't need to store your pictures in
a format specific to {\TC} and then ``export'' to a \LaTeX\, source in
another file. In addition, it allows to rework a picture made with
another program, e.g. GNUPLOT.

\subsection{Integrating a {\TC} picture into a \LaTeX\, file}
\label{insert}
All you have is to include your {\TC} file (which is also just \LaTeX\, code)
into your text, either via the {\tt \bs input} command, or by inserting
it directly.
%
For instance, you save your picture under the name {\tt toto.pic},
and in \LaTeX, put {\tt \bs input toto.pic} where you want the picture
to appear, or, alternatively, open {\tt toto.pic} in a text editor, copy
the full contents, paste it at that place.\\
For example, the following insertion:
\\
\begin{center}
\begin{tabular}{|c|}
\hline
{\tiny
\begin{minipage}[t]{0.7\linewidth}
\begin{verbatim}
For obvious continuity reasons, the interpolation curve {\em will}
go outside the range if the slope has the inverse sign of the difference
of values (this excludes the cases with a zero slope or zero delta):
%
\begin{center}
%
%TeXCAD Picture [u3.tcp]. Options:
%\grade{\off}
%\emlines{\off}
%\epic{\off}
%\beziermacro{\on}
%\reduce{\on}
%\snapping{\on}
%\quality{8.00}
%\graddiff{0.01}
%\snapasp{1}
%\zoom{4.0000}
\unitlength 1.2pt
\linethickness{0.4pt}
\ifx\plotpoint\undefined\newsavebox{\plotpoint}\fi % GNUPLOT compatibility
\begin{picture}(81,41)(0,0)
\put(10,10){\framebox(70,30)[cc]{}}
\put(10,10){\circle*{2}}
\put(80,40){\circle*{2}}
\put(2,12){\line(4,-1){40}}
\qbezier(10,10)(59,-3)(80,40)
\end{picture}
%
\end{center}
%
\end{verbatim}
\end{minipage}
}\\
\hline
\end{tabular}
\end{center}
%
will look like:
\begin{center}
\begin{tabular}{|c|}
\hline
{\tiny
\begin{minipage}[t]{0.7\linewidth}
For obvious continuity reasons, the interpolation curve {\em will}
go outside the range if the slope has the inverse sign of the difference
of values (this excludes the cases with a zero slope or zero delta):
%
\begin{center}
%
%TeXCAD Picture [u3.tcp]. Options:
%\grade{\off}
%\emlines{\off}
%\epic{\off}
%\beziermacro{\on}
%\reduce{\on}
%\snapping{\on}
%\quality{8.00}
%\graddiff{0.01}
%\snapasp{1}
%\zoom{4.0000}
\unitlength 1.2pt
\linethickness{0.4pt}
\ifx\plotpoint\undefined\newsavebox{\plotpoint}\fi % GNUPLOT compatibility
\begin{picture}(81,41)(0,0)
\put(10,10){\framebox(70,30)[cc]{}}
\put(10,10){\circle*{2}}
\put(80,40){\circle*{2}}
\put(2,12){\line(4,-1){40}}
\qbezier(10,10)(59,-3)(80,40)
\end{picture}
%
\end{center}
%
\end{minipage}
}\\
\hline
\end{tabular}
\end{center}
%
See the FAQ (\ref{faq}) about safely removing the \TeX\, comments from
the picture code.
%
\subsection{Compatibility} If there are commands unknown to {\TC},
for instance color selection or whatever,
they are {\bf preserved and stored in the same place} of the
picture file when {\TC} saves the picture.
Note that this program is upward-compatible with the previous DOS
``incarnation'', {\TC} 3.2.
%
\section{How to use {{\TC}} on MS Windows}
%
The program behaves like usual Windows programs, so we assume
you have a basic knowledge in that subject...
%
\subsection{Installing and uninstalling}
If you are a fan of watching ads during the hours an installation
program may take, you'll be disappointed: {\TC} doesn't need it.
The program is wholly contained in {\TC}.exe --- not even a {\tt DLL} around!
So, if you rather dislike installations with lots of files scattered across
your hard drive and Windows' system area, you'll be happy.
You can even put {\TC}.exe in a read-only drive, for instance a network
drive without writing rights for users, since {\TC} doesn't write
any configuration file. Since the user settings are stored
in the registry (HKEY\_CURRENT\_USER\bs Software\bs TeXCAD), the
desinstallation requires running Uninstall\_TeXCAD.bat
logged as a super-user (in that case all {\TC} information will be removed).
As a ``normal" user, you will just remove your own settings by this way.
%
\subsection{Outlines}
{\TC} is a multi-document program, so you can work on several
drawings --- appearing each in a sub-window ---, copy things from
one to the clipboard, the paste them from the clipboard to another
window.

The {\bf mouse buttons} are defined so --- this will be familiar to those who have used the DOS' {\TC} --- :
\begin{itemize}
\item The left button is for drawing or picking objects.
This is the ``Yes'' or ``+'' button.
\item The right button is for cancelling operations or unpicking objects
This is the ``No'' or ``-'' button.
\end{itemize}

Another DOS' {\TC} feature: you can move the mouse cursor with the arrow keys, e.g. for fine tuning.

We describe hereafter the functionalities of {\TC} for Windows in the
way the menus appear.
%
\subsection{Files}
{\TC} is, like almost all common programs, file-based: there is
a 1-to-1 relation between the picture being worked and a file name
on a certain drive --- with the exception of a new, unsaved
picture which doesn't correspond to a file until it is saved the first
time. Each time you save the picture, the file contains exactly
the information currently displayed.

The commands ``New", ``Open", ``Save", ``Close", etc. should sound
familiar to the user. On the menu's bottom there is a list of recently
opened files.

\subsubsection{Preview}
The ``Preview" command compiles the current picture with LaTeX and
calls the installed DVI previewer - known to me are YAP, and
\href{http://dviwin.keystone.gr/}{DVIWin}
which has currently a much better rendering.

If you have trouble with previewing, here is how it
functions more in detail: {\TC} writes,
in the temporary directory \footnote{Alternatively, you can choose in
the general options the {\bf current} directory instead, but {\it caution}, the
notion of ``current'' directory is floating if you have several pictures
open stored in different places!},
a {\tt TeXCADpv.tex} file
as well as a {\tt TeXCADpv.bat} with the lines :
\begin{verbatim}
  call latex TeXCADpv.tex
  start TeXCADpv.dvi
\end{verbatim}
So, if you don't have the ``latex" command available (in the PATH) you
can create a batch file ``latex.bat" with all workarounds you like...

If text boxes in your picture refers to specific definitions, you
can set in the \LaTeX \, tab in the picture options these
definitions. The format is free, you can put there whatever you
want, it will be inserted between the \verb+\documentclass+ line
and the \verb+\begin{document}+ line.

NB: as from version 4.2 of \TC, the preview uses by default the
\verb+\documentclass+ syntax (\LaTeX\, $2\epsilon$ or later) in {\tt
TeXCADpv.tex} instead of the 2.09 \verb+\documentstyle+ syntax.
You can still use the old syntax by selecting the appropriate
general option.

\subsection{Drawing}
\subsubsection{Objects defined by 1 point}
It is a text in a broad sense. It can be also formulas between \$ ...
\$, a sub-picture, an embedded EPS image, or whatever you put in
your \LaTeX\, document. Text is included in a {\tt \bs makebox}
with alignment options or in a raw {\tt \bs put} command.
You need one click for choosing the spot where the text appears.
%
\subsubsection{Objects defined by 2 points}
\begin{itemize}
\item {\bf Rectangles}. You can choose between a frame
or a filled rectangle. For non filled rectangles you can put a
text in it --- a dialog box will appear for entering it and set
alignment.
\item {\bf Lines and vectors}. The function is obvious. In the
picture options you can choose to have only the slope choice of
the {\tt \bs line} and {\tt \bs vector} commands of \LaTeX\, or
any slope. {\TC} finds out how to draw these line, independently
of the choice of add-ons (none, emlines, pstricks,...).
\item {\bf Circles and discs}. These functions are quite obvious,
aren't they ? Note that circles of a radius more than 20pt are
not supported by pure \LaTeX, but anyway {\TC}, even in the mode
without add-ons, is able to draw them with small line segments. You
can choose their quality in the option panel. Same for discs: the \LaTeX\,
command is valid only up to 7.5pt; above that, {\TC} fills the disc
fractally with boxes.
\item {\bf Ovals} These are rather rounded rectangles. You can
choose which side or corner will appear.
\end{itemize}
%
All figures determined by two points need drawing the mouse:
press left button on first end and release it on the other end.
Lines and vectors can also be chained: the further segments need clicking
--- in short: draw, click, click,... right-click to stop.
%
\subsubsection{Objects defined by 3 points}
%
For the moment, there are only B\'ezier splines in that category.
B\'ezier curves are made by clicking the three points determining them,
plus two per further curve
--- in short: click, click, click, click, click,... right-click to
stop.
Initially the splines are filled. You can change it with the
``Change text or parameters'' (\ref{chgtxt}) menu command and clicking on the
curve. In the general option panel, you can choose whether you want
the filled curves first determined by an amount of points fixed by {\TC} or let \LaTeX\, find it, but this requires {\tt \bs qbezier} ---
included for long in \LaTeX\, --- instead of the old {\tt \bs
bezier}.
%
\subsection{Lines (in a broad sense: also curves)}
From {\TC} 4.1 the line settings has been made ``orthogonal" to the figures
themselves, hence a new menu and the disappearance of some redundant
figure choices. Some combinations are not meaningful
(like arrows for boxes), or yet programmed in {\TC}.
At the end of this paragraph there is a synopsis of currently possible combinations.
%
\subsubsection{Thin or thick}
``thin" relates to the {\tt \bs thinlines} macro (just the normal thickness)
and ``thick" to {\tt \bs thicklines}, which displays lines (and curves)
with a double thickness.
% In some cases \LaTeX doesn't seem to display effectively with a double
% thickness. No extensive list of these cases, sorry!
%
\subsubsection{Patterns}
You can choose and configure the following patterns:
plain, dotted, dashed. Note that when you choose to recognize
the {\tt epic} environment, {\TC} makes use of its
{\tt \bs dottedline} and {\tt \bs dashline} macros.
%
\subsubsection{Arrows}
%
You have there the choice between ``no arrow" (will output e.g. {\tt \bs line}
or {\tt \bs bezier}), ``head" (e.g. {\tt \bs vector}), ``both", arrow at both ends,
``middle", an arrow on middle.
%
\subsubsection{Synopsis of combinations figures / line settings}
%
\begin{center}
{\scriptsize
{\renewcommand{\arraystretch}{1.2}
{\tabcolsep = 2.2pt
\begin{tabular}{|l r|c|c|c|c|} \hline
\multicolumn{1}{|l}{\bf Line}&
\multicolumn{1}{|r|}
{\begin{tabular}{c c} & \framebox{pattern}\\ \framebox{arrows}\end{tabular}}
&
\multicolumn{1}{c}{\sf plain, limited slope}&
\multicolumn{1}{c}{\sf plain, any slope}&
\multicolumn{1}{c}{\sf dotted}&
\multicolumn{1}{c|}{\sf dash}\\
\cline{2-6}
&{\sf no\_ arrow}&
 {$\backslash$}line&
 {\begin{tabular}{c}
  {$\backslash$}drawline\\
  {$\backslash$}emline, {\%}{$\backslash$}emline
  \end{tabular}
 }
 &
 {$\backslash$}dottedline&
 {$\backslash$}dashline\\ \cline{3-6}
&{\sf head}&
 {$\backslash$}vector&
 {\%}{$\backslash$}vector&
 {\%}{$\backslash$}vector{$\lbrace$}dot{$\rbrace$}&
 {\%}{$\backslash$}vector{$\lbrace$}dash{$\rbrace$}\\ \cline{3-6}
&{\sf both}&
 {\%}{$\backslash$}vector{\lbrack}b{\rbrack}{$\lbrace$}{$\backslash$}line{$\rbrace$}&
 {\%}{$\backslash$}vector{\lbrack}b{\rbrack}&
 {\%}{$\backslash$}vector{\lbrack}b{\rbrack}{$\lbrace$}dot{$\rbrace$}&
 {\%}{$\backslash$}vector{\lbrack}b{\rbrack}{$\lbrace$}dash{$\rbrace$}\\ \cline{3-6}
&{\sf middle}&
 {\%}{$\backslash$}vector{\lbrack}m{\rbrack}{$\lbrace$}{$\backslash$}line{$\rbrace$}&
 {\%}{$\backslash$}vector{\lbrack}m{\rbrack}&
 {\%}{$\backslash$}vector{\lbrack}m{\rbrack}{$\lbrace$}dot{$\rbrace$}&
 {\%}{$\backslash$}vector{\lbrack}m{\rbrack}{$\lbrace$}dash{$\rbrace$}\\ \hline
\multicolumn{5}{c}{}\\
\cline{1-5}
\multicolumn{1}{|l}{\bf Box}&
\multicolumn{1}{|r|}{\framebox{pattern}}&
\multicolumn{1}{c}{\sf plain}&
\multicolumn{1}{c}{\sf dotted}&
\multicolumn{1}{c|}{\sf dash}\\
\cline{2-5}
\multicolumn{1}{|c}{}&&{$\backslash$}framebox&{\%}{$\backslash$}dottedbox&{$\backslash$}dashbox\\
\cline{1-5}
\multicolumn{6}{c}{}\\
\cline{1-3}
\multicolumn{1}{|l}{\bf Bezier}&\multicolumn{1}{|l|}{\framebox{arrows}}    &
&
\multicolumn{3}{c}{}\\
\cline{2-3}
&{\sf no\_ arrow}&{$\backslash$}{\lbrack}{\lbrack}q{\rbrack}{\rbrack}bezier&\multicolumn{3}{c}{}\\
\cline{3-3}
&{\sf head}&{\%}{$\backslash$}bezvec&\multicolumn{3}{c}{}\\ \cline{3-3}
&{\sf both}&{\%}{$\backslash$}bezvec{\lbrack}b{\rbrack}&\multicolumn{3}{c}{}\\
\cline{3-3}
&{\sf middle}&{\%}{$\backslash$}bezvec{\lbrack}m{\rbrack}&\multicolumn{3}{c}{}\\
\cline{1-3}
\end{tabular}
} %% fin tabcolsep
} %% fin arraystretch
}
\end{center}
%
The commands preceded by a `\verb+%+' are {\em not}\, \LaTeX \, commands
(or of the supported and switched on extensions like {\tt epic}), but are to be understood by \TC only.
The code used by \LaTeX \, is in the lines between such a command (the command itself is
seen by \LaTeX \, as a comment) and the line with \verb+%\end+.
See the FAQ (\S\ref{faq}) Nr \ref{tccommands} for an example.
%
\subsection{Editing}
\subsubsection{Change text or parameters}
\label{chgtxt}
%
After having selected this menu entry, you can pick certain object
for changing their contents and/or parameters:\\
\begin{center}
\begin{tabular}{|l|l|}
\hline
Object          & Content or parameter to change\\
\hline
Text            & The text itself, alignment\\
Rectangle       & Text, text alignment, dot length\\
Oval            & Displayed edges or corners\\
B\'ezier curves & Number of dots, or automatic\\
\hline
\end{tabular}
\end{center}
%
\subsubsection{Picking objects (individually, in an area, or all)}
%
This how you start picking objects ({\sl Pick objects}) in order
to modify, copy or deleting them.
\begin{itemize}
\item
If you click on an object with
the left mouse button, you select (``pick'') it, {\bf individually}.
If you click on an already selected object
with the right button, you ``unpick'' it.
The matching criterion is the distance of mouse cursor to the object
(for a box: the distance to the frame or to the attached text).
%
\item
You can also select objects in an {\bf area} by pushing
the mouse button when the mouse cursor is in one corner of the area rectangle
and releasing it on the opposite corner. Inbetween you see a
dotted frame corresponding to the area's border.
Same for un-selecting an area: you do the press-move-release with the right button.
The matching criterion is the presence of the {\em full} object in the
area dotted frame.
%
\item
Finally you can {\sl Select all } (the)
{\sl objects}, including the hidden ones {\TC} didn't understand at loading,
or {\sl Unselect everything} of which was eventually selected previously.
\end{itemize}
%
\subsubsection{Geometric operations}
Once there are ``picked'' items, you can apply them
a translation, a rotation, a symmetry (with a certain choice
of axes) or an affine transformation. The translation needs
a vector, you draw it as if it was a picture object.
For the others a point (i.e. a mouse click) is needed.
%
\subsubsection{Delete, cut, copy, save macro}
%
All these operations need naturally that you have
selected previously at least one object.
%
\subsubsection{Paste, load macro}
%
{\sl Paste} and {\sl load macro} operate on the same
fashion: after choosing the menu item, you are asked to click on
the spot where the lower left corner of the virtual rectangle
containing the imported objects will land onto your picture.
%
\subsection{View, toolbars}
The ``View" menu allows to choose which toolbar is shown.
The ``Drawing tools" toolbar corresponds to the ``Drawing" menu,
the ``Line settings" toolbar to the ``Line" menu.
%
\subsection{Options}
%
There are {\sl General options} that contain various parameters you
--- as user --- permanently prefer, and {\sl Picture options} that
are specific to the picture of the currently active subwindow.
Note that you can in {\sl General options} choose default parameters
for every new picture.
In the {\sl Picture options} the compatibility ones need maybe
some explanation. As the \LaTeX\, world evolves some extensions
to \{picture\} become obsolete, like emlines, other become almost
standard. ``Almost'' is always the delicate point, since every
\LaTeX\, user or group of users has its own combination of
add-ons, packages, fixes, patches, previewers, converters, devices
and operating systems. For instance the PostScript-related
add-ons are ``almost'' standard, but they are far of working
in {\em every} situation, even if a PostScript printer is at the
end of the line. Of course each package is a genial piece of work
and I won't contest the quality of any. Simply, the philosophy
behind {\TC} is: if you like such or such package, use it, but
we don't force you to use it. We always find a solution without it,
at the price of a bigger picture file. After all the idea of
\TeX\, is to be DeVice Independent...
%
%
\section{Frequently Asked Questions (FAQ)} \label{faq}
%
\begin{enumerate}
%
\item
{\em How can I put {\TC} picture in my \LaTeX\, file?}

It's straightforward: just include or insert it - see \S \ref{insert}.

\item
{\em Do I have to add a special style file? }

Only the ones you choose with the Picture Option panel
(See ``Compatibility (.sty)'').


\item
\label{tccommands}
{\em {\TC} puts plenty of useless comments in the file it writes. Why ?}

Either they were already in the file as input, or they correspond
to a command for {\TC} which is not supported by a standard \LaTeX\, macro
or by a macro of an extension recognized by {\TC} and enabled in the picture options.

The ``active'' \LaTeX\, code is between such a command (like: \verb+%\dottedbox+)
and a line
consisting of: \verb+%\end+. Please do {\bf not} remove these comments!
To get rid of them, you can activate a corresponding {\tt .sty} option
(picture options), or avoid using the corresponding figure (e.g.
dotted lines if you don't want to use {\tt epic}). Here is an example that shows
how the principle works:

{\bf With} {\tt epic} enabled you get:
  \begin{center}\footnotesize{
    \begin{tabular}{|c|}
    \hline
    \begin{minipage}[t]{0.75\linewidth}
    \begin{verbatim}
    \dottedline(7,8.25)(45.25,19.5)
    \end{verbatim}
    \end{minipage}\\
    \hline
    \end{tabular}
  }\end{center}
{\bf Without} {\tt epic}, {\TC} provides an emulation
for \LaTeX\, but considers itself only the code
in the first commented line:
  \begin{center}\footnotesize{
    \begin{tabular}{|c|}
    \hline
    \begin{minipage}[t]{0.75\linewidth}
    \begin{verbatim}
    %\dottedline(7,8.25)(45.25,19.5)
    \multiput(6.93,8.18)(.93293,.27439){42}{{\rule{.4pt}{.4pt}}}
    %\end
    \end{verbatim}
    \end{minipage}\\
    \hline
    \end{tabular}
  }\end{center}

Here is a more subtle case which shows that {\TC} is able to
provide several levels of emulation according to which style
extensions are enabled:
\begin{enumerate}
\item Nothing, pure \LaTeX : a B\'ezier curve is drawn with sequence of small
segments; since the slopes of pure \LaTeX\, are limited, some
oblique segments are drawn via an emulation with a sequence of small horizontal
segments:
  \begin{center}\tiny{
    \begin{tabular}{|c|}
    \hline
    \begin{minipage}[t]{0.75\linewidth}
    \begin{verbatim}
%\qbezier(2,1.5)(4.563,2.438)(5.875,3.125)
\multiput(2,1.5)(.12527,.046929){8}{\line(1,0){.12527}}
\multiput(3.002,1.875)(.108492,.042532){6}{\line(1,0){.108492}}
\multiput(3.653,2.131)(.099481,.04073){6}{\line(1,0){.099481}}
\multiput(4.25,2.375)(.10965,.04693){4}{\line(1,0){.10965}}
\multiput(4.689,2.563)(.10208,.04542){3}{\line(1,0){.10208}}
\multiput(4.995,2.699)(.09559,.04412){3}{\line(1,0){.09559}}
\put(5.282,2.831){\line(1,0){.1804}}
\put(5.462,2.917){\line(1,0){.1717}}
\put(5.634,3.002){\line(1,0){.1631}}
\put(5.797,3.084){\line(1,0){.0783}}
%\end
    \end{verbatim}
    \end{minipage}\\
    \hline
    \end{tabular}
  }\end{center}
\item {\tt epic} enabled but not {\tt bezier}: the sequence of segments emulating the curve
uses the \verb+\drawline+ command in {\tt epic}:
  \begin{center}\tiny{
    \begin{tabular}{|c|}
    \hline
    \begin{minipage}[t]{0.75\linewidth}
    \begin{verbatim}
%\qbezier(2,1.5)(4.563,2.438)(5.875,3.125)
\drawline(2,1.5)(3.002,1.875)(3.653,2.131)(4.25,2.375)(4.689,2.563)(4.995,2.699)
(5.282,2.831)(5.462,2.917)(5.634,3.002)(5.797,3.084)(5.875,3.125)
%\end
    \end{verbatim}
    \end{minipage}\\
    \hline
    \end{tabular}
  }\end{center}
\item {\tt bezier} enabled, other options meaningless:
  \begin{center}\footnotesize{
    \begin{tabular}{|c|}
    \hline
    \begin{minipage}[t]{0.75\linewidth}
    \begin{verbatim}
\qbezier(2,1.5)(4.563,2.438)(5.875,3.125)
    \end{verbatim}
    \end{minipage}\\
    \hline
    \end{tabular}
  }\end{center}

\end{enumerate}
%
As a conclusion for this question, please don't remove the \TeX\, comments
inside the {\TC} picture text, unless you intend not to rework your
picture again anymore. Even then, you will be able to do it but won't see
everything on screen, or lose some structures or groupings. Note that you can
remove harmlessly the comments before ``\verb#\begin{picture}#''
since they only contain the switches for the {\TC} picture options.
%
\end{enumerate}
%
\section{The authors}
%
\begin{tabular}{l c l}
Georg Horn, J\"orn Winkelmann &:& The original {\TC} for DOS (1989-1994),\\
&& in Pascal, up to v. 3.2\\
Gautier de Montmollin &:& Translation to Ada
via
\href{http://p2ada.sf.net}{P2Ada},\\
&&{\TC} 4.x system,\\
&&Windows ``hull''\\
\end{tabular}\\[1em]

I (Gautier) am the maintainer of the current project.\\
Contact me at \hrefid{http://sourceforge.net/users/gdemont/}.

The latest version is (or might be) on the following URL:\\
\hrefid{http://texcad.sf.net}.\\
The project is also hosted at SourceForge, URL:\\
\hrefid{http://sourceforge.net/projects/texcad/}.
%
\section{Acknowledgements --- for tools, help, testing, ideas,
remarks, asking questions}
%
Note that all wishes are not yet concretized. Be patient, it is a
only question of time! So, thanks to:
%
\begin{itemize}
  \item David Botton, for his GNAVI R.A.D. system, in
        particular his wonderful GWindows library (URL:
        \hrefid{http://sourceforge.net/projects/gnavi/})
        and his help to this project.
  \item
  \begin{itemize}
    \item
        Prof. Robert A.G. Seely (McGill), for his enthusiastic testing.
    \item
        Bernd S. W. Schroeder (Louisiana Tech University), for suggestions
        and his review in \href{http://www.merlot.org/}{MERLOT}.
  \end{itemize}
  \item Also thanks to \ldots
    \begin{itemize}
    \item {\em User side:}
        Th. Bauer,
        Prof. Dr.-Ing. Helmut Bollenbacher,
        Prof. Carlos A. Cinquetti,
        Prof. Jean-Pierre Corriou,
        Maxwell Dondo,
        Zeno Enders,
        Roman Heinisch,
        Shul-John Li,
        Alessandro Rosa,
        Paolo Tommasini,
        Rune T\o nnesen,
        Song Yu,
        Dr Yahya H. Zweiri.
    \item {\em Programming side:}
        Andr\'e van Splunter
  \end{itemize}
\end{itemize}
%
\section{How to build the program yourself}
%

To build the sources, you need an Ada compiler, for instance the
GNU Ada compiler (GNAT) that can be found there:
  \hrefid{http://libre.adacore.com/} \\
Alternatively you can use the MinGW version: \hrefid{http://mingw.org/}.
In case there is an implementation for Linux, GNAT is available with
the standard packages.
  
The project uses extensively Ada's modularity.
The platform-dependent parts are confined to their minimum and
the core of {\TC} is totally portable across platforms and
compilers, without conditional compilation -- which doesn't exist in Ada
by the way! The core includes:
%
\begin{itemize}
  \item the memory storage of pictures
  \item transformations
  \item input / output
  \item object picking rules
  \item \ldots and even the display, which is possible in a platform-independent
        way through the magic of Ada's {\em generic programming}
\end{itemize}
%
As a test I build and run regularily the core with GNAT, on both Windows
and Linux, as well as with another development system,
ObjectAda 7.2.2 SE (free version), available at \hrefid{http://www.aonix.com/}.\\

The Windows ``skin'' uses the open-source framework GWindows for the user interface.

All you need to do for building {\TC} for Windows is to install 
a binary distribution of the {\bf GNAT Ada compiler} for Windows platforms
(from AdaCore or MinGW, links above).

Then, go to the TC\_GWin directory and run {\tt "build\_debug.cmd"},
or open {\tt texcad\_gwin.gpr} with GPS (the GNAT Programming Studio) and press F4.

NB: a snapshot of the {\bf GWindows} library, ANSI mode, is included with
{\TC}'s sources (in {\tt TC\_GWin/windows\_stuff/}), so you don't need to install it!
If for whatever reason you need to update GWindows, you'll find it there:
  \hrefid{http://sourceforge.net/projects/gnavi/}
%
$$ $$
The principal directories are:\\
%
\begin{tabular}{|l c l|}\hline
  (root) &:& the core of {\TC} as described earlier.
             There is a test program,\\
         & & Test\_TC\_IO, mainly an entropy test,
             that does a massive number of\\
         & & Load/Format-change/Save operations
             and checks if something gets\\
         & & lost, also in accuracy.\\
  TC\_GWin&:& the parts for the Windows version.\\
  Test\_IO&:& a few pictures, some of mine, some grabbed from the web,\\
           &&some output of GNUPlot. Test\_TC\_IO reads them and write\\
           &&copies, for checking consistency.\\ \hline
\end{tabular}
%
$$ $$
The full tree looks like:
%
\begin{center}
{\small
%
%TeXCAD Picture [folders.tcp]. Options:
%\grade{\on}
%\emlines{\off}
%\epic{\off}
%\beziermacro{\on}
%\reduce{\on}
%\snapping{\on}
%\pvinsert{\def\TC{{\TeX}CAD}}
%\quality{8.00}
%\graddiff{0.01}
%\snapasp{1}
%\zoom{6.7272}
\unitlength 3pt
\linethickness{0.4pt}
\ifx\plotpoint\undefined\newsavebox{\plotpoint}\fi % GNUPLOT compatibility
\begin{picture}(133,51)(0,0)
\put(8,48){\line(0,-1){5}}
\put(8,43){\line(1,0){7}}
\put(15,43){\line(0,1){5}}
\put(15,48){\line(-1,0){7}}
%\emline(8,48)(9,49)
\multiput(8,48)(.03125,.03125){32}{\line(0,1){.03125}}
%\end
\put(9,49){\line(1,0){2}}
%\emline(11,49)(12,48)
\multiput(11,49)(.03125,-.03125){32}{\line(0,-1){.03125}}
%\end
\put(18,40){\line(0,-1){5}}
\put(18,35){\line(1,0){7}}
\put(25,35){\line(0,1){5}}
\put(25,40){\line(-1,0){7}}
%\emline(18,40)(19,41)
\multiput(18,40)(.03125,.03125){32}{\line(0,1){.03125}}
%\end
\put(19,41){\line(1,0){2}}
%\emline(21,41)(22,40)
\multiput(21,41)(.03125,-.03125){32}{\line(0,-1){.03125}}
%\end
\put(11,37){\line(1,0){7}}
\put(21,29){\line(1,0){7}}
\put(28,32){\line(0,-1){5}}
\put(28,27){\line(1,0){7}}
\put(35,27){\line(0,1){5}}
\put(35,32){\line(-1,0){7}}
%\emline(28,32)(29,33)
\multiput(28,32)(.03125,.03125){32}{\line(0,1){.03125}}
%\end
\put(29,33){\line(1,0){2}}
%\emline(31,33)(32,32)
\multiput(31,33)(.03125,-.03125){32}{\line(0,-1){.03125}}
%\end
\put(18,24){\line(0,-1){5}}
\put(18,19){\line(1,0){7}}
\put(25,19){\line(0,1){5}}
\put(25,24){\line(-1,0){7}}
%\emline(18,24)(19,25)
\multiput(18,24)(.03125,.03125){32}{\line(0,1){.03125}}
%\end
\put(19,25){\line(1,0){2}}
%\emline(21,25)(22,24)
\multiput(21,25)(.03125,-.03125){32}{\line(0,-1){.03125}}
%\end
\put(18,16){\line(0,-1){5}}
\put(18,8){\line(0,-1){5}}
\put(18,11){\line(1,0){7}}
\put(18,3){\line(1,0){7}}
\put(25,11){\line(0,1){5}}
\put(25,3){\line(0,1){5}}
\put(25,16){\line(-1,0){7}}
\put(25,8){\line(-1,0){7}}
%\emline(18,16)(19,17)
\multiput(18,16)(.03125,.03125){32}{\line(0,1){.03125}}
%\end
%\emline(18,8)(19,9)
\multiput(18,8)(.03125,.03125){32}{\line(0,1){.03125}}
%\end
\put(19,17){\line(1,0){2}}
\put(19,9){\line(1,0){2}}
%\emline(21,17)(22,16)
\multiput(21,17)(.03125,-.03125){32}{\line(0,-1){.03125}}
%\end
%\emline(21,9)(22,8)
\multiput(21,9)(.03125,-.03125){32}{\line(0,-1){.03125}}
%\end
\put(11,21){\line(1,0){7}}
\put(11,13){\line(1,0){7}}
\put(11,5){\line(1,0){7}}
\put(21,34){\line(0,-1){5}}
\put(17,46){\makebox(0,0)[lc]{(root)}}
\put(27,38){\makebox(0,0)[lc]{TC\_GWin}}
\put(37,30){\makebox(0,0)[lc]{Windows\_Stuff}}
\put(11,42){\line(0,-1){37}}
\put(27,22){\makebox(0,0)[lc]{Test\_IO}}
\put(27,14){\makebox(0,0)[lc]{ACU\_Opti}}
\put(27,6){\makebox(0,0)[lc]{ACU\_Debg}}
%\dottedline{1}(26,45)(66,45)
\multiput(25.93,44.93)(.97561,0){42}{{\rule{.4pt}{.4pt}}}
%\end
%\dottedline{1}(40,37)(66,37)
\multiput(39.93,36.93)(.963,0){28}{{\rule{.4pt}{.4pt}}}
%\end
%\dottedline{1}(54,29)(66,29)
\multiput(53.93,28.93)(.9231,0){14}{{\rule{.4pt}{.4pt}}}
%\end
%\dottedline{1}(37,21)(66,21)
\multiput(36.93,20.93)(.9667,0){31}{{\rule{.4pt}{.4pt}}}
%\end
%\dottedline{1}(40,13)(66,13)
\multiput(39.93,12.93)(.963,0){28}{{\rule{.4pt}{.4pt}}}
%\end
%\dottedline{1}(40,5)(66,5)
\multiput(39.93,4.93)(.963,0){28}{{\rule{.4pt}{.4pt}}}
%\end
\put(67,45){\makebox(0,0)[lb]{Core of {\TC}}}
\put(67,37){\makebox(0,0)[lb]{Parts for the Windows version}}
\put(67,29){\makebox(0,0)[lb]{GWindows, GNATCOM frameworks}}
\put(67,21){\makebox(0,0)[lb]{Test pictures}}
\put(67,13){\makebox(0,0)[lb]{Compiled parts of the optimized version}}
\put(67,5){\makebox(0,0)[lb]{Compiled parts of the debug version}}
\put(69,26){\oval(128,50)[]}
\end{picture}
%
}
\end{center}
%
\end{document}
